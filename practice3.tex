\documentclass[a4paper]{article}
\usepackage{xcolor}
\usepackage{graphicx}
\usepackage{multirow}
\usepackage{amsmath}
\usepackage[margin=0.5in]{geometry}
\title{\textbf{\textcolor{red}{Rajiv Gandhi University Of knowledge Technologies\\Internal Assignment-2}}}
\author{Computer Science and engineering\\IT workshop}
\date{November 22,2021}
\begin{document}
\maketitle
\vspace{1mm}
\hrule
\vspace{2mm}
\textsl{Newton's Universal Law of gravitation}
\begin{center}
\begin{eqnarray}
F=G(\frac{m1*m2}{d2})
\end{eqnarray}
\begin{eqnarray}
\lim\limits_{x \to \infty}exp(x)=0
\end{eqnarray}
\colorbox{yellow}{$log_a(xy)=log_a(x)+log_a(y)$}
\begin{eqnarray}
\left (\int x^ndx=\frac{x^{(n+1)}}{n+1}+c\right)
\end{eqnarray}
\end{center}
\begin{flushleft}
\vspace{2mm}
\begin{tabular}{|c|c|c|}
\hline
\multicolumn{3}{|c|}{\textbf{Semester grade card}}\\
\hline
subjectName & Marks & grade\\
\hline
Itworkshop & 90 & Ex\\
\hline
C programming & 80& A\\
\hline
Java & 75 & B\\
\hline
\end{tabular}
\end{flushleft}
\begin{center}
\large{\textbf{National Symbols of India}}\\
\vspace{2mm}
\includegraphics[scale=0.5]{india}
\end{center}
\end{document}