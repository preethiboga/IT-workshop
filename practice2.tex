\documentclass{article}
\usepackage{multirow}
\usepackage{xcolor}
\usepackage{graphicx}
\usepackage[margin=0.5in]{geometry}
\usepackage{amsmath}
\begin{document}
\hrule
\vspace{2mm}
\begin{center}
\textbf{\huge{Homework{\LARGE{\#}}1}}\\
\vspace{1mm}
$^1$  \footnotesize{Student name:Felipe Portales-oliva}\\
\vspace{2mm}
\hrule
\vspace{2mm}
\textbf{\footnotesize{Course:Special Realtivity(physics 301)-Proffesor: Dr.Albert Einstein}}\\
\vspace{1mm}
Due date: March 28th,2005\\
\vspace{1mm}
\colorbox{blue}{\textcolor{white}{Rajiv Gandhi University of Knowledge Technologies}}\\
\vspace{1mm}
\textbf{Question 1}\\
\begin{tabular}{|c c|}
\hline
what is the air speed velocity of an unladen swallow?&\hspace{8mm}\\
\hline
\end{tabular}\\
\vspace{3mm}
\includegraphics[scale=0.3]{cat}
\end{center}
\vspace{2mm}
We have now added a title,author and date to our first LATEX documet!\\
Some of the greatest \textit{discoveries} in science were made by accident.\textsl{some of the greatest discoveries in\\ science were made by accident.}
\textbf{Some greatest \textit{discoveries} in science were made by accident.}\\
\begin{center}
\begin{itemize}
\item The individual entries are indicated with a black dot, a so-called bullet.
\item The text in the entries may be of any length
\end{itemize}
\begin{enumerate}
\item This is the first entry in our list
\item This list numbers increase with each entry we add
\end{enumerate}
\end{center}
\begin{flushleft}
The mass-energy equivalence is described by the famous equation\\
\begin{center}
\vspace{2mm}
$E=mc^2$
\end{center}
discovered in 1905 by Albert Einstein.
\end{flushleft}
\begin{center}
\begin{tabular}{|c c c|}
\hline
cell1 & cell2 & cell3\\
cell4 & cell5 & cell6\\
cell7 & cell8 & cell9\\
\hline
\end{tabular}
\end{center}
\begin{flushleft}
We write integrals using $\int$ and fractions using $\frac{a}{b}$.Limits are placed on integrals using\\superscripts and subscripts
\end{flushleft}
\begin{center}
$\int_0  ^1e^\frac{dx}{x}=\frac{e-1}{e}$
\end{center}
\begin{flushleft}
\section{Introduction}
\section{Second section}
\subsection{First subsection}
\subsubsection{second Section}
\end{flushleft}
\footnote{funded by the RGUKT}
\end{document}